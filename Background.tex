\section{Background}
% no \IEEEPARstart
%% Olfactory perception description

%% Previous work on olfactory perception

%% Previous work on emotion-related olfactory perception

%% Paper structure


%\hfill mds
 
%\hfill January 11, 2007



% An example of a floating figure using the graphicx package.
% Note that \label must occur AFTER (or within) \caption.
% For figures, \caption should occur after the \includegraphics.
% Note that IEEEtran v1.7 and later has special internal code that
% is designed to preserve the operation of \label within \caption
% even when the captionsoff option is in effect. However, because
% of issues like this, it may be the safest practice to put all your
% \label just after \caption rather than within \caption{}.
%
% Reminder: the "draftcls" or "draftclsnofoot", not "draft", class
% option should be used if it is desired that the figures are to be
% displayed while in draft mode.
%
%\begin{figure}[!t]
%\centering
%\includegraphics[width=2.5in]{myfigure}
% where an .eps filename suffix will be assumed under latex, 
% and a .pdf suffix will be assumed for pdflatex; or what has been declared
% via \DeclareGraphicsExtensions.
%\caption{Simulation Results}
%\label{fig_sim}
%\end{figure}

% Note that IEEE typically puts floats only at the top, even when this
% results in a large percentage of a column being occupied by floats.


% An example of a double column floating figure using two subfigures.
% (The subfig.sty package must be loaded for this to work.)
% The subfigure \label commands are set within each subfloat command, the
% \label for the overall figure must come after \caption.
% \hfil must be used as a separator to get equal spacing.
% The subfigure.sty package works much the same way, except \subfigure is
% used instead of \subfloat.
%
%\begin{figure*}[!t]
%\centerline{\subfloat[Case I]\includegraphics[width=2.5in]{subfigcase1}%
%\label{fig_first_case}}
%\hfil
%\subfloat[Case II]{\includegraphics[width=2.5in]{subfigcase2}%
%\label{fig_second_case}}}
%\caption{Simulation results}
%\label{fig_sim}
%\end{figure*}
%
% Note that often IEEE papers with subfigures do not employ subfigure
% captions (using the optional argument to \subfloat), but instead will
% reference/describe all of them (a), (b), etc., within the main caption.


% An example of a floating table. Note that, for IEEE style tables, the 
% \caption command should come BEFORE the table. Table text will default to
% \footnotesize as IEEE normally uses this smaller font for tables.
% The \label must come after \caption as always.
%
%\begin{table}[!t]
%% increase table row spacing, adjust to taste
%\renewcommand{\arraystretch}{1.3}
% if using array.sty, it might be a good idea to tweak the value of
% \extrarowheight as needed to properly center the text within the cells
%\caption{An Example of a Table}
%\label{table_example}
%\centering
%% Some packages, such as MDW tools, offer better commands for making tables
%% than the plain LaTeX2e tabular which is used here.
%\begin{tabular}{|c||c|}
%\hline
%One & Two\\
%\hline
%Three & Four\\
%\hline
%\end{tabular}
%\end{table}


% Note that IEEE does not put floats in the very first column - or typically
% anywhere on the first page for that matter. Also, in-text middle ("here")
% positioning is not used. Most IEEE journals/conferences use top floats
% exclusively. Note that, LaTeX2e, unlike IEEE journals/conferences, places
% footnotes above bottom floats. This can be corrected via the \fnbelowfloat
% command of the stfloats package.

% AFFECTIVE COMPUTING GENERAL IDEA

% PRIMARY RESPONSE TO ODORS, LINKS WITH EMOTIONS AND AFFECTIVE COMPUTING
%Emotion is involved in every aspect of human life, thus, it has gained attention in many research disciplines, including computer science. For instance, 
\subsection{Functional Connectivity}
Brain connectivity refers to a pattern of anatomical links (anatomical connectivity) or of statistical dependencies (functional connectivity) between neural assemblies. The connectivity pattern is formed by structural links such as synapses or represented by statistical or causal relationships measured as cross-correlation, coherence or information flow~\cite{sporns2007brain}. Brain connectivity is a crucial concept to elucidate how neural networks process information. A neurophysiological concept of functional connectivity is introduced by A.A Fingelkurts~\cite{fingelkurts2005functional}. According to Fingelkurts' concept, functional connectivity is described as the mechanism for the coordination of activity between different neural assemblies in order to achieve a complex cognitive task or perceptual process. However, since the dependence between different sub-regions of the brain can be interpreted in different ways, there exists different models to estimate functional connectivity in the brain. Widely used models include time-domain estimations such as \emph{Linear Granger Causality}~\cite{granger1969investigating}, \emph{Nonlinear Regression Analysis}~\cite{pijn1990localization}, \emph{Transfer Entropy}~\cite{schreiber2000measuring} and frequency-domain estimations such as \emph{Spectral Coherence}~\cite{sun2004measuring}.

The concept of functional connectivity has been largely used in neuro-imaging analysis in the study of major depression~\cite{greicius2007resting} and epilepsy~\cite{waites2006functional} to study the information flow patterns during seizures. To our knowledge, little work has been done on emotion recognization on olfactory perception using functional connectivity patterns. Thus in this paper, we introduce the concept of functional connectivity to investigate the connctivity patterns between brain regions during odor pleasantness perception. We applied and compared two time-domain functional connectivity models: \emph{Linear Granger Causality} and \emph{Nonlinear Regression Analysis}. By comparing the performances of the two models, we can better understand the characteristics of connections between brain regions during olfactory perception and emotion generation. The differences of connectivity patterns between pleasant and unpleasant ordor perceptions can be generalized to other emotion study and also help us process or stimulate human affects in the future work. 

\subsection{Network Features}
The functional connectivity gives us a view of how neural assemblies communicate information with each other, thus we can consider the whole connectivities over brain as a kind of brain network. For an N-channel EEG signal recording, functional connectivity estimates the connectivity between each channel combination, which results in an $N \times N$ size connectivity network. With the increase of number of channels $N$, the brain network will become larger and more difficult to be analyzed directly. Thus, we proposed to extract network-based features from such brain network. The extracted network-based features from original functional connectivity over brain can provide a higher-dimension view on the characteristics of the connectivity networks. 

Some research groups see the brain network as a kind of small-world network~\cite{bassett2006small} while some others view it as a scale-free network~\cite{eguiluz2005scale}. A small-world network is a type of network whose node can be reached from every others by a small number of steps even when they are not neighbors. A scale-free newtork is a kind of network most of whose nodes have only a few connections, while only a small number of nodes have a lot of connections. The degree distribution of a scale-free network follows a power law~\cite{stam2004functional}. 

Both scale-free network and small-world network can provide features based on their own characteristics. Small-world network can provide features as \emph{characteristic path}~\cite{watts1998collective}, \emph{local and global efficiency}, \emph{clustering coefficient}~\cite{latora2001efficient}. Scale-free network can provide features as \emph{shannon entropy}~\cite{shannon2001mathematical} and \emph{von Neumann entropy}~\cite{passerini2008neumann}. In this paper, we extracted both types of features for training the classifiers on odor pleasantness. The performances of the classifiers can help us understand which type of network the functional connectivity network can be considered as. 

