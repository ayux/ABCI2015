\section{Background}
% no \IEEEPARstart
%% Olfactory perception description

%% Previous work on olfactory perception

%% Previous work on emotion-related olfactory perception

%% Paper structure


%\hfill mds
 
%\hfill January 11, 2007



% An example of a floating figure using the graphicx package.
% Note that \label must occur AFTER (or within) \caption.
% For figures, \caption should occur after the \includegraphics.
% Note that IEEEtran v1.7 and later has special internal code that
% is designed to preserve the operation of \label within \caption
% even when the captionsoff option is in effect. However, because
% of issues like this, it may be the safest practice to put all your
% \label just after \caption rather than within \caption{}.
%
% Reminder: the "draftcls" or "draftclsnofoot", not "draft", class
% option should be used if it is desired that the figures are to be
% displayed while in draft mode.
%
%\begin{figure}[!t]
%\centering
%\includegraphics[width=2.5in]{myfigure}
% where an .eps filename suffix will be assumed under latex, 
% and a .pdf suffix will be assumed for pdflatex; or what has been declared
% via \DeclareGraphicsExtensions.
%\caption{Simulation Results}
%\label{fig_sim}
%\end{figure}

% Note that IEEE typically puts floats only at the top, even when this
% results in a large percentage of a column being occupied by floats.


% An example of a double column floating figure using two subfigures.
% (The subfig.sty package must be loaded for this to work.)
% The subfigure \label commands are set within each subfloat command, the
% \label for the overall figure must come after \caption.
% \hfil must be used as a separator to get equal spacing.
% The subfigure.sty package works much the same way, except \subfigure is
% used instead of \subfloat.
%
%\begin{figure*}[!t]
%\centerline{\subfloat[Case I]\includegraphics[width=2.5in]{subfigcase1}%
%\label{fig_first_case}}
%\hfil
%\subfloat[Case II]{\includegraphics[width=2.5in]{subfigcase2}%
%\label{fig_second_case}}}
%\caption{Simulation results}
%\label{fig_sim}
%\end{figure*}
%
% Note that often IEEE papers with subfigures do not employ subfigure
% captions (using the optional argument to \subfloat), but instead will
% reference/describe all of them (a), (b), etc., within the main caption.


% An example of a floating table. Note that, for IEEE style tables, the 
% \caption command should come BEFORE the table. Table text will default to
% \footnotesize as IEEE normally uses this smaller font for tables.
% The \label must come after \caption as always.
%
%\begin{table}[!t]
%% increase table row spacing, adjust to taste
%\renewcommand{\arraystretch}{1.3}
% if using array.sty, it might be a good idea to tweak the value of
% \extrarowheight as needed to properly center the text within the cells
%\caption{An Example of a Table}
%\label{table_example}
%\centering
%% Some packages, such as MDW tools, offer better commands for making tables
%% than the plain LaTeX2e tabular which is used here.
%\begin{tabular}{|c||c|}
%\hline
%One & Two\\
%\hline
%Three & Four\\
%\hline
%\end{tabular}
%\end{table}


% Note that IEEE does not put floats in the very first column - or typically
% anywhere on the first page for that matter. Also, in-text middle ("here")
% positioning is not used. Most IEEE journals/conferences use top floats
% exclusively. Note that, LaTeX2e, unlike IEEE journals/conferences, places
% footnotes above bottom floats. This can be corrected via the \fnbelowfloat
% command of the stfloats package.

% AFFECTIVE COMPUTING GENERAL IDEA

% PRIMARY RESPONSE TO ODORS, LINKS WITH EMOTIONS AND AFFECTIVE COMPUTING
%Emotion is involved in every aspect of human life, thus, it has gained attention in many research disciplines, including computer science. For instance, 
Multimedia systems are increasingly becoming immersive, in order to evoke strong emotions and render user-experience more realistic. Traditionally, multimedia systems include video and audio contents, they, thus, mainly stimulate the visual and auditory senses. Nevertheless, recently odors have started to be incorporated into multimedia systems (e.g., \cite{nakamoto2011olfactory,nakamoto2008cooking,richard2006multi}), since they directly stimulate memories and elicit strong emotions. However, emotion elicitation from odors has not been adequately investigated, although the primary response to smell is related to pleasantness perception \cite{gulas1995right}. 

% RESEARCHER'S INVESTIGATIONS ON PLEASANTNESS
Perception of pleasantness from various stimuli has been investigated by various researchers through different means. Many studies have been conducted on the investigation of pleasantness perception through facial expressions~\cite{lyons1998coding}, food intake\cite{de2003taste}, languages~\cite{bellezza1986words}, etc. Research on pleasantness detection and classification has been carried out by analyzing brain activity using various brain imaging techniques (e.g., \cite{zatorre2000neural,kringelbach2003activation,kroupi2014eeg}).

% PREVIOUS WORKS/PAPERS ON ANALYSIS FROM PSD, BANDS FEATURES
Although pleasantness perception has been thoroughly analysed for various, especially audiovisual, stimuli, it has received less attention during experience of odors. Moreover, various electroencephalography (EEG) studies on investigating odor pleasantness have analysed brain activation in terms of power spectral density features (e.g., \cite{kroupi2012multivariate,joussain2013effect,kroupi2014eeg}), but further information from EEG could contribute to a better understanding of pleasantness perception from odors. For instance, a question that still remains unanswered is regarding the way in which odor pleasantness influences functional connectivity patterns in remote brain locations. The hypothesis is that there are differences in the functional connectivity patterns when subjects experience pleasant and when they experience unpleasant odors. However, it is still unclear what type of network the brain can be seen as during olfactory perception, and if there are network-based features able to discriminate pleasant and unpleasant odors. 

%The emotion status during odor perception still need to be investigated for a better understanding the olfactory perception and emotion generation processes. The current studies on classifying emotional responses from odors based on EEG signals are more focused on extracting features of power spectral density, workload index (WI), permutation entropy, dimension of minimal covers~\cite{kroupi2014eeg}. 

% VALUE OF FUNCTIONAL CONNECTIVITY OF EEG
%Most of the methods that have been proposed and commonly used in EEG-based emotion study are performed directly on EEG signals themselves, while in this study we propose to use an indirect method to study about these EEG signals which is called \emph{functional connectivity map}. 
%In general, there are different levels of connectivity in the brain, ranging from the connections between neighbouring neurons to the connections between different segregated brain regions. The concept of functional connectivity describes the dependence between the activities of different neuron assemblies. The functional connectivity map describes the dependency between all the sub-regions of brain (all the electrodes in EEG signals) regardless of whether they are structurally linked or not. 
The concept of functional connectivity maps has been largely used in neuro-imaging analysis to study various disorders with respect to control states, such as for instance major depression~\cite{greicius2007resting} and epilepsy~\cite{waites2006functional}. Based on the neurophysiological principle~\cite{van2010exploring}, brain can be viewed as a complex network. Thus in this paper, we introduce the concept of functional connectivity to investigate the networking of the brain during odor pleasantness perception.

% OUR METHODS, NO RESULTS
Since the dependence between different sub-regions of the brain can be interpreted in different ways, there exist different models of estimating functional connectivity in the brain. Widely used models include \emph{Granger Causality}~\cite{granger1969investigating}, \emph{Transfer Entropy}~\cite{schreiber2000measuring}, \emph{Nonlinear Regression Analysis}~\cite{pijn1990localization} and \emph{Spectral Coherence}~\cite{sun2004measuring}. In this paper, we apply and compare Granger causality and nonlinear regression analysis. In order to quantify differences in the estimated networks, we extract network-based features such as \emph{characteristic path}~\cite{watts1998collective}, \emph{local and global efficiency}, \emph{clustering coefficient}~\cite{latora2001efficient}, \emph{shannon entropy} and \emph{von Neumann entropy}~\cite{passerini2008neumann}. The extracted features are then used for classification of perceived odor pleasantness. 

% STRUCTURE OF THE PAPER
The rest of the paper is organised as follows: Section II describes the experimental protocols and methods for constructing and measuring functional connectivity maps from EEG signals. Section III provides the classification results on pleasantness by using network-based features extracted from the functional connectivity maps. Section IV gives the conclusions of this work. 