\section{Conclusion}
The concept of functional connectivity maps has been used to study the brain activity but it has not yet been used for classifying odor pleasantness perception. In this paper, we compared different methods of estimating functional connectivity from EEG signals for 23 subjects in order to classify pleasant and unpleasant odors. By considering the connectivity maps as networks, physical statistics and graph theory based features were extracted and used in SVM classifiers. The best classification accuracy based on Cohen's Kappa was achieved using nonlinear regression analysis and small-world network features to estimate the connectivity maps. The results indicated that nonlinear patterns occur in the connectivity maps during hedonic olfactory perception, able to classify odor pleasantness perception in a significantly non-random way.  
